\documentclass{article}
\usepackage[utf8]{inputenc}

\begin{document}

\section*{Chapter 1 Questions}

\subsection*{1.1}
\begin{enumerate}
    \item[1.1] [2] <§1.1> Aside from the smart cell phones used by a billion people, list and describe four other types of computers.
\end{enumerate}

\subsection*{1.2}
\begin{enumerate}
    \item[1.2] [5] <§1.2> The eight great ideas in computer architecture are similar to ideas from other fields. Match the eight ideas from computer architecture, "Design for Moore's Law," "Use Abstraction to Simplify Design," "Make the Common Case Fast," "Performance via Parallelism," "Performance via Pipelining," "Performance via Prediction," "Hierarchy of Memories," and "Dependability via Redundancy" to the following ideas from other fields:
    \begin{enumerate}
        \item[a.] Assembly lines in automobile manufacturing
        \item[b.] Suspension bridge cables
        \item[c.] Aircraft and marine navigation systems that incorporate wind information
        \item[d.] Express elevators in buildings
        \item[e.] Library reserve desk
        \item[f.] Increasing the gate area on a CMOS transistor to decrease its switching time
        \item[g.] Adding electromagnetic aircraft catapults (which are electrically powered as opposed to current steam-powered models), allowed by the increased power generation offered by the new reactor technology
        \item[h.] Building self-driving cars whose control systems partially rely on existing sensor systems already installed into the base vehicle, such as lane departure systems and smart cruise control systems
    \end{enumerate}
\end{enumerate}

\subsection*{1.3}
\begin{enumerate}
    \item[1.3] [2] <§1.3> Describe the steps that transform a program written in a high-level language such as C into a representation that is directly executed by a computer processor.
\end{enumerate}

\subsection*{1.4}
\begin{enumerate}
    \item[1.4] [2] <§1.4> Assume a color display using 8 bits for each of the primary colors (red, green, blue) per pixel and a frame size of 1280 × 1024.
    \begin{enumerate}
        \item[a.] What is the minimum size in bytes of the frame buffer to store a frame?
        \item[b.] How long would it take, at a minimum, for the frame to be sent over a 100Mbit/s network?
    \end{enumerate}
\end{enumerate}

\subsection*{1.5}
\begin{enumerate}
    \item[1.5] [4] <§1.6> Consider three different processors P1, P2, and P3 executing the same instruction set. P1 has a 3GHz clock rate and a CPI of 1.5. P2 has a 2.5GHz clock rate and a CPI of 1.0. P3 has a 4.0GHz clock rate and has a CPI of 2.2.
    \begin{enumerate}
        \item[a.] Which processor has the highest performance expressed in instructions per second?
        \item[b.] If the processors each execute a program in 10 seconds, find the number of cycles and the number of instructions.
        \item[c.] We are trying to reduce the execution time by 30% but this is proving difficult because each of the three processors has a single-cycle clock rate. How much do we have to reduce the CPI and clock rate to achieve this 30% reduction? Justify your answer.
    \end{enumerate}
\end{enumerate}

