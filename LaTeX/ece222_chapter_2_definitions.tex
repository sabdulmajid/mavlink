\documentclass{article}
\begin{document}
\begin{enumerate}
    \item \textbf{Instruction set:} The vocabulary of commands understood by a given architecture.
    \item \textbf{Stored-program concept:} The idea that instructions and data of many types can be stored in memory as numbers and thus be easy to change, leading to the stored-program computer.
    \item \textbf{Doubleword:} Another natural unit of access in a computer, usually a group of 64 bits; corresponds to the size of a register in the RISC-V architecture.
    \item \textbf{Word:} A natural unit of access in a computer, usually a group of 32 bits.
    \item \textbf{Data transfer instruction:} A command that moves data between memory and registers.
    \item \textbf{Address:} A value used to delineate the location of a specific data element within a memory array.
    \item \textbf{Alignment restriction:} A requirement that data be aligned in memory on natural boundaries.
    \item \textbf{Binary digit:} Also called bit. One of the two numbers in base 2, 0 or 1, that are the components of information.
    \item \textbf{Least significant bit:} The rightmost bit in an RISC-V doubleword.
    \item \textbf{Most significant bit:} The leftmost bit in an RISC-V doubleword.
    \item \textbf{One's complement:} A notation that represents the most negative value by $10\cdots000_2$ and the most positive value by $01\cdots11_2$, leaving an equal number of negatives and positives but ending up with two zeros, one positive ($00\cdots00_2$) and one negative ($11\cdots11_2$). The term is also used to mean the inversion of every bit in a pattern: 0 to 1 and 1 to 0.
    \item \textbf{Biased notation:} A notation that represents the most negative value by $00\cdots000_2$ and the most positive value by $11\cdots11_2$, with 0 typically having the value $10\cdots00_2$, thereby biasing the number such that the number plus the bias has a non-negative representation.
    \item \textbf{Instruction format:} A form of representation of an instruction composed of fields of binary numbers.
    \item \textbf{Machine language:} Binary representation used for communication within a computer system.
    \item \textbf{Hexadecimal:} Numbers in base 16.
    \item \textbf{Opcode:} The field that denotes the operation and format of an instruction.
    \item \textbf{Basic block:} A sequence of instructions without branches (except possibly at the end) and without branch targets or branch labels (except possibly at the beginning).
    \item \textbf{Branch address table:} Also called branch table. A table of addresses of alternative instruction sequences.
    \item \textbf{Procedure:} A stored subroutine that performs a specific task based on the parameters with which it is provided.
    \item \textbf{Jump-and-link instruction:} An instruction that branches to an address and simultaneously saves the address of the following instruction in a register (usually x1 in RISC-V).
    \item \textbf{Return address:} A link to the calling site that allows a procedure to return to the proper address; in RISC-V it is stored in register x1.
    \item \textbf{Caller:} The program that instigates a procedure and provides the necessary parameter values.
    \item \textbf{Callee:} A procedure that executes a series of stored instructions based on parameters provided by the caller and then returns control to the caller.
    \item \textbf{Program counter (PC):} The register containing the address of the instruction in the program being executed.
    \item \textbf{Stack:} A data structure for spilling registers organized as a last-in-first-out queue.
    \item \textbf{Stack pointer:} A value denoting the most recently allocated address in a stack that shows where registers should be spilled or where old register values can be found. In RISC-V, it is register sp, or x2.
    \item \textbf{Global pointer:} The register that is reserved to point to the static area.
    \item \textbf{Procedure frame:} Also called activation record. The segment of the stack containing a procedure’s saved registers and local variables.
    \item \textbf{Frame pointer:} A value denoting the location of the saved registers and local variables for a given procedure.
    \item \textbf{Text segment:} The segment of a UNIX object file that contains the machine language code for routines in the source file.
    \item \textbf{PC-relative addressing:} An addressing regime in which the address is the sum of the program counter (PC) and a constant in the instruction.
    \item \textbf{Addressing mode:} One of several addressing regimes delimited by their varied use of operands and/or addresses.
    \item \textbf{Data race:} Two memory accesses form a data race if they are from different threads to the same location, at least one is a write, and they occur one after another.
    \item \textbf{Assembly language:} A symbolic language that can be translated into binary machine language.
    \item \textbf{Pseudoinstruction:} A common variation of assembly language instructions often treated as if it were an instruction in its own right.
    \item \textbf{Symbol table:} A table that matches names of labels to the addresses of the memory words that instructions occupy.
    \item \textbf{Linker:} Also called link editor. A systems program that combines independently assembled machine language programs and resolves all undefined labels into an executable file.
    \item \textbf{Executable file:} A functional program in the format of an object file that contains no unresolved references. It can contain symbol tables and debugging information. A "stripped executable" does not contain that information. Relocation information may be included for the loader.
    \item \textbf{Loader:} A systems program that places an object program in main memory so that it is ready to execute.
    \item \textbf{Dynamically linked libraries (DLLs):} Library routines that are linked to a program during execution.
    \item \textbf{Java bytecode:} Instruction from an instruction set designed to interpret Java programs.
    \item \textbf{Java Virtual Machine (JVM):} The program that interprets Java bytecodes.
    \item \textbf{Just In Time compiler (JIT):} The name commonly given to a compiler that operates at runtime, translating the interpreted code segments into the native code of the computer.
\end{enumerate}
\end{document}
