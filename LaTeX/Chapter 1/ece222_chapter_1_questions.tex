\documentclass[11pt]{article}
\usepackage[utf8]{inputenc}
\usepackage[margin=0.57in]{geometry}

\begin{document}

\section*{Chapter 1 Questions - Assignment Questions for Week 1}

% \subsection*{1.1}
% \begin{enumerate}
%     \item[1.1] [2] <§1.1> Aside from the smart cell phones used by a billion people, list and describe four other types of computers.
% \end{enumerate}

% \subsection*{1.2}
% \begin{enumerate}
%     \item[1.2] [5] <§1.2> The eight great ideas in computer architecture are similar to ideas from other fields. Match the eight ideas from computer architecture, "Design for Moore's Law," "Use Abstraction to Simplify Design," "Make the Common Case Fast," "Performance via Parallelism," "Performance via Pipelining," "Performance via Prediction," "Hierarchy of Memories," and "Dependability via Redundancy" to the following ideas from other fields:
%     \begin{enumerate}
%         \item[a.] Assembly lines in automobile manufacturing
%         \item[b.] Suspension bridge cables
%         \item[c.] Aircraft and marine navigation systems that incorporate wind information
%         \item[d.] Express elevators in buildings
%         \item[e.] Library reserve desk
%         \item[f.] Increasing the gate area on a CMOS transistor to decrease its switching time
%         \item[g.] Adding electromagnetic aircraft catapults (which are electrically powered as opposed to current steam-powered models), allowed by the increased power generation offered by the new reactor technology
%         \item[h.] Building self-driving cars whose control systems partially rely on existing sensor systems already installed into the base vehicle, such as lane departure systems and smart cruise control systems
%     \end{enumerate}
% \end{enumerate}

% \subsection*{1.3}
% \begin{enumerate}
%     \item[1.3] [2] <§1.3> Describe the steps that transform a program written in a high-level language such as C into a representation that is directly executed by a computer processor.
% \end{enumerate}

\begin{enumerate}
    \item[\textbf{1.4}] \textbf{<§1.4>} Assume a color display using 8 bits for each of the primary colors (red, green, blue) per pixel and a frame size of 1280 × 1024.
    \begin{enumerate}
        \item[a.] What is the minimum size in bytes of the frame buffer to store a frame?
        \item[b.] How long would it take, at a minimum, for the frame to be sent over a 100Mbit/s network?
    \end{enumerate}
\end{enumerate}


% \begin{enumerate}
%     \item[1.5] [4] <§1.6> Consider three different processors P1, P2, and P3 executing the same instruction set. P1 has a 3GHz clock rate and a CPI of 1.5. P2 has a 2.5GHz clock rate and a CPI of 1.0. P3 has a 4.0GHz clock rate and has a CPI of 2.2.
%     \begin{enumerate}
%         \item[a.] Which processor has the highest performance expressed in instructions per second?
%         \item[b.] If the processors each execute a program in 10 seconds, find the number of cycles and the number of instructions.
%         \item[c.] We are trying to reduce the execution time by 30\% but this is proving difficult because each of the three processors has a single-cycle clock rate. How much do we have to reduce the CPI and clock rate to achieve this 30\% reduction? Justify your answer.
%     \end{enumerate}
% \end{enumerate}

\begin{enumerate}
    \item[\textbf{1.6}] \textbf{<§1.6>} A given program runs on two different computer systems. The first computer system completes the execution of the program in 10 seconds, while the second computer system completes the execution of the program in 15 seconds.
    \begin{enumerate}
        \item[a.] Calculate the speedup ratio of the first computer system compared to the second computer system
        \item[b.] Calculate the performance ratio of the first computer system compared to the second computer system
        \item[c.] If the execution time of the program on the first computer system is reduced by 20\% while keeping the execution time on the second computer system the same, calculate the new speedup ratio and performance ratio
    \end{enumerate}
\end{enumerate}

\begin{enumerate}
    \item[\textbf{1.7}] \textbf{<§1.7>} Consider a system with a clock frequency of 2.5GHz and an average CPI of 1.5. When running a program, 35\% of the instructions are arithmetic, 20\% are load/store, 25\% are branch, and 20\% are other instructions. The CPI for each instruction category is as follows: 1.0 for arithmetic, 1.5 for load/store, 2.0 for branch, and 3.0 for other instructions
    \begin{enumerate}
        \item[a.] Calculate the effective CPI for the program.
        \item[b.] If the program executes 1 billion instructions, calculate the total execution time.
    \end{enumerate}
\end{enumerate}


\begin{enumerate}
    \item[\textbf{1.8}] \textbf{<§1.8>} A computer system has a 32-bit virtual address and a 4 KB page size. It uses a two-level page table where the first-level page table contains 1024 entries, and each entry is of size 4 bytes. The second-level page table contains 512 entries, and each entry is of size 4 bytes. Each page table entry also has a valid bit. Calculate the total memory overhead required for the page table of a process.
\end{enumerate}

% \subsection*{1.9}
% \begin{enumerate}
%     \item[1.9] [3] <§1.10> A computer system has a cache with a capacity of 256 KB and a block size of 64 bytes. The address generated by the processor is 32 bits. Calculate the number of index bits, offset bits, and tag bits in the address format used by the cache.
% \end{enumerate}


\begin{enumerate}
    \item[\textbf{1.10}] \textbf{<§1.10>} A computer system has a level-1 data cache with a hit time of 1 ns, a miss time of 10 ns, and a hit rate of 90\%. It also has a level-2 data cache with a hit time of 10 ns, a miss time of 100 ns, and a hit rate of 80\%. Calculate the average memory access time (AMAT) for the system.
\end{enumerate}


\begin{enumerate}
    \item[\textbf{1.11}] \textbf{<§1.11>} A computer system uses a 32-bit virtual address and a 4 KB page size. The system has a translation lookaside buffer (TLB) with a hit rate of 95\% and a hit time of 1 ns. The TLB is small enough to hold 32 entries. The system also has a page table stored in memory with an access time of 100 ns. Calculate the average memory access time (AMAT) for the system.
\end{enumerate}


\begin{enumerate}
    \item[\textbf{1.12}] \textbf{<§1.12>} A computer system has a cache with a hit time of 1 ns, a miss time of 10 ns, and a hit rate of 90\%. The system also has a main memory with an access time of 100 ns. The hit rate of the cache can be improved to 95\% by adding a small buffer between the cache and the main memory. This buffer introduces a delay of 5 ns on a cache miss, but it allows the cache to access the main memory in parallel with the buffer. Calculate the average memory access time (AMAT) for the system with the buffer.
\end{enumerate}


\begin{enumerate}
    \item[\textbf{1.13}] \textbf{<§1.13>} A computer system has a cache with a miss penalty of 20 ns and a miss rate of 5\%. The cache is connected to a main memory with an access time of 100 ns. A processor running a program generates 100 memory references, out of which 40\% are loads and 60\% are stores. Calculate the average memory access time (AMAT) for the system.
\end{enumerate}

\end{document}
