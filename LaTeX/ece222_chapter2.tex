\documentclass{article}

\begin{document}

\section*{ECE222 Chapter 2 Notes}

\subsection*{Instruction Set Architecture}
\begin{itemize}
    \item Instruction set architecture (ISA) defines the interface between hardware and software.
    \item RISC-V instructions are 32-bits (instruction[31:0]).
    \item RISC-V assembly uses 64-bit registers (double word) and 32-bit word registers.
    \item There are 32 registers (x0-x31), with x0 always being zero.
    \item Arithmetic operations require data to be in registers.
    \item 'Less frequently used' variables are 'spilled' into memory.
    \item Registers are faster and more energy efficient than memory.
    \item RISC-V has a 16-bit instruction set (RISC-V compressed) for embedded applications with code size constraints.
    \item Memory is byte-addressable with little endian byte ordering.
    \item RISC-V uses a Harvard architecture with separate instruction and data caches.
\end{itemize}

\subsection*{Instruction Formats}
\begin{itemize}
    \item R-type instructions: Use three register operands (2 sources and 1 destination).
    \item I-type instructions: Use two register operands and a 12-bit immediate value.
    \item S-type instructions: Use two register operands and a 12-bit immediate value for stores.
    \item SB-type instructions: Conditional branch instructions with PC-relative addressing.
    \item U-type instructions: Upper immediate format for adding immediate value to PC.
    \item UJ-type instructions: Unconditional jump instructions for jumps and links.
\end{itemize}

\subsection*{Examples}
\begin{itemize}
    \item R-type instruction example: \texttt{add x5,x6,x2} is coded as \texttt{0000000 00010 00110 000 00101 0110011}.
    \item I-type instruction example: \texttt{ld x9, 64(x22)} is coded as \texttt{0000 0100 0000 10110 011 01001 0000011}.
    \item S-type instruction example: \texttt{sd x9, 240(x10)} is coded as \texttt{0000111 01001 01010 011 10000 0100011}.
    \item SB-type instruction example: \texttt{bne x10, x11, 2000} is coded as \texttt{1100000 01011 01010 001 11110 1100011}.
    \item U-type instruction example: \texttt{auipc x1, 1000} is coded as \texttt{0000000 10001 00000 11011 0010111}.
    \item UJ-type instruction example: \texttt{jal x11, 2000} is coded as \texttt{1000000 01011 00000 11011 1101111}.
\end{itemize}

\section{Code Examples}

\subsection{Compilers and Branches}
Compilers frequently create branches and labels where they do not appear in the programming language.

\subsection{RISC-V Logical Operations}
RISC-V logical operations include shift left/right, right arithmetic, bitwise AND/OR/NOT/XOR.

\subsubsection{Example 1}
\begin{verbatim}
if (i == j)
    f = g + h;
else
    f = g - h;

assume f through j correspond to values in registers x19 through x23

bne x22, x23, Else  // go to Else if i != j
add x19, x20, x21  // f = g + h
beq x0, x0, Exit   // go to Exit

Else: sub x19, x20, x21  // f = g - h

Exit:
\end{verbatim}

\subsubsection{Example 2}
\begin{verbatim}
while (save[i] == k)
    i += 1;

assume value of i is in register x22 and value of k is in register x24.
Assume base of array save[] is in x25.

loop: slli x10, x22, 3      // x10 = i * 8
       add x10, x10, x25    // x10 now has address of save[]
       ld x9, 0(x10)        // load save[i] into x9
       bne x9, x24, Exit    // if save[i] != k go to exit
       addi x22, x22, 1     // i++
       beq x0, x0, Loop     // go to loop

Exit:
\end{verbatim}

\subsection{Procedures}
Procedures allow programmers to concentrate on just one portion of the task at a time.

\subsubsection{Procedure Instructions}
By convention in RISC-V:
\begin{itemize}
    \item 8 parameter registers x10-x17 are used to pass parameters or return values.
    \item One return address register x1 holds the return address to return to the point of origin.
\end{itemize}

In RISC-V, there are special procedure instructions:
\begin{itemize}
    \item Jump and link instruction, jal (jump and link, UJ type) for procedures, which branches to an address and saves the address of the following instruction (PC+4) to rd=x1.
\end{itemize}


\end{document}
