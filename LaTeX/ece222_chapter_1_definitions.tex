\begin{document}
\begin{enumerate}
    \item \textbf{Personal computer (PC):} A computer designed for use by an individual, usually incorporating a graphics display, a keyboard, and a mouse.
    
    \item \textbf{Server:} A computer used for running larger programs for multiple users, often simultaneously, and typically accessed only via a network.
    
    \item \textbf{Supercomputer:} A class of computers with the highest performance and cost; they are configured as servers and typically cost tens to hundreds of millions of dollars.
    
    \item \textbf{Terabyte (TB):} Originally 1,099,511,627,776 (240) bytes, although communications and secondary storage systems developers started using the term to mean 1,000,000,000,000 (1012) bytes. To reduce confusion, we now use the term tebibyte (TiB) for 240 bytes, defining terabyte (TB) to mean 1012 bytes. Figure 1.1 shows the full range of decimal and binary values and names.
    
    \item \textbf{Embedded computer:} A computer inside another device used for running one predetermined application or collection of software.
    
    \item \textbf{Personal mobile devices (PMDs):} Small wireless devices to connect to the Internet; they rely on batteries for power, and software is installed by downloading apps. Conventional examples are smartphones and tablets.
    
    \item \textbf{Cloud Computing:} Large collections of servers that provide services over the Internet; some providers rent dynamically varying numbers of servers as a utility.
    
    \item \textbf{Software as a Service (SaaS):} Delivers software and data as a service over the Internet, usually via a thin program such as a browser that runs on local client devices, instead of binary code that must be installed, and runs wholly on that device. Examples include web search and social networking.
    
    \item \textbf{Systems software:} Software that provides services that are commonly useful, including operating systems, compilers, loaders, and assemblers.
    
    \item \textbf{Operating system:} Supervising program that manages the resources of a computer for the benefit of the programs that run on that computer.
    
    \item \textbf{Compiler:} A program that translates high-level language statements into assembly language statements.
    
    \item \textbf{Binary digit:} Also called a bit. One of the two numbers in base 2 (0 or 1) that are the components of information.
    
    \item \textbf{Instruction:} A command that computer hardware understands and obeys.
    
    \item \textbf{Assembler:} A program that translates a symbolic version of instructions into the binary version.
    
    \item \textbf{Assembly language:} A symbolic representation of machine instructions.
    
    \item \textbf{Machine language:} A binary representation of machine instructions.
    
    \item \textbf{Input device:} A mechanism through which the computer is fed information, such as a keyboard.
    
    \item \textbf{Output device:} A mechanism that conveys the result of a computation to a user, such as a display, or to another computer.
    
    \item \textbf{Liquid crystal display (LCD):} A display technology using a thin layer of liquid polymers that can be used to transmit or block light according to whether a charge is applied.
    
    \item \textbf{Active matrix display:} A liquid crystal display using a transistor to control the transmission of light at each individual pixel.
    
    \item \textbf{Pixel:} The smallest individual picture element. Screens are composed of hundreds of thousands to millions of pixels, organized in a matrix.
    
    \item \textbf{Integrated circuit:} Also called a chip. A device combining dozens to millions of transistors.
    
    \item \textbf{Central processing unit (CPU):} Also called processor. The active part of the computer, which contains the datapath and control and which adds numbers, tests numbers, signals I/O devices to activate, and so on.
    
    \item \textbf{Datapath:} The component of the processor that performs arithmetic operations.
    
    \item \textbf{Control:} The component of the processor that commands the datapath, memory, and I/O devices according to the instructions of the program.
    
    \item \textbf{Memory:} The storage area in which programs are kept when they are running and that contains the data needed by the running programs.
    
    \item \textbf{Dynamic random access memory (DRAM):} Memory built as an integrated circuit; it provides random access to any location. Access times are 50 nanoseconds and cost per gigabyte in 2012 was \$5 to \$10.
    
    \item \textbf{Cache memory:} A small, fast memory that acts as a buffer for a slower, larger memory.
    
    \item \textbf{Static random access memory (SRAM):} Memory built as an integrated circuit, but faster and less dense than DRAM.
    
    \item \textbf{Instruction set architecture:} Also called architecture. An abstract interface between the hardware and the lowest-level software that encompasses all the information necessary to write a machine language program that will run correctly, including instructions, registers, memory access, I/O, and so on.
    
    \item \textbf{Application binary interface (ABI):} The user portion of the instruction set plus the operating system interfaces used by application programmers. It defines a standard for binary portability across computers.
    
    \item \textbf{Implementation:} Hardware that obeys the architecture abstraction.
    
    \item \textbf{Volatile memory:} Storage, such as DRAM, that retains data only if it is receiving power.
    
    \item \textbf{Nonvolatile memory:} A form of memory that retains data even in the absence of a power source and that is used to store programs between runs. A DVD disk is nonvolatile.
    
    \item \textbf{Main memory:} Also called primary memory. Memory used to hold programs while they are running; typically consists of DRAM in today's computers.
    
    \item \textbf{Secondary memory:} Nonvolatile memory used to store programs and data between runs; typically consists of flash memory in PMDs and magnetic disks in servers.
    
    \item \textbf{Magnetic disk:} Also called hard disk. A form of nonvolatile secondary memory composed of rotating platters coated with a magnetic recording material. Because they are rotating mechanical devices, access times are about 5 to 20 milliseconds and cost per gigabyte in 2012 was \$0.05 to \$0.10.
    
    \item \textbf{Flash memory:} A nonvolatile semiconductor memory. It is cheaper and slower than DRAM but more expensive per bit and faster than magnetic disks. Access times are about 5 to 50 microseconds and cost per gigabyte in 2012 was \$0.75 to \$1.00.
    
    \item \textbf{Local area network (LAN):} A network designed to carry data within a geographically confined area, typically within a single building.
    
    \item \textbf{Wide area network (WAN):} A network extended over hundreds of kilometers that can span a continent.
    
    \item \textbf{Transistor:} An on/off switch controlled by an electric signal.
    
    \item \textbf{Very large-scale integrated (VLSI) circuit:} A device containing hundreds of thousands to millions of transistors.
    
    \item \textbf{Silicon:} A natural element that is a semiconductor.
    
    \item \textbf{Semiconductor:} A substance that does not conduct electricity well.
    
    \item \textbf{Silicon crystal ingot:} A rod composed of a silicon crystal that is between 8 and 12 inches in diameter and about 12 to 24 inches long.
    
    \item \textbf{Wafer:} A slice from a silicon ingot no more than 0.1 inches thick, used to create chips.
    
    \item \textbf{Die:} The individual rectangular sections that are cut from a wafer, more informally known as chips.
    
    \item \textbf{Yield:} The percentage of good dies from the total number of dies on the wafer.
    
    \item \textbf{Response time:} Also called execution time. The total time required for the computer to complete a task, including disk accesses, memory accesses, I/O activities, operating system overhead, CPU execution time, and so on.
    
    \item \textbf{Throughput:} Also called bandwidth. Another measure of performance, it is the number of tasks completed per unit time.
    
    \item \textbf{CPU execution time:} Also called CPU time. The actual time the CPU spends computing for a specific task.
    
    \item \textbf{User CPU time:} The CPU time spent in a program itself.
    
    \item \textbf{System CPU time:} The CPU time spent in the operating system performing tasks on behalf of the program.
    
    \item \textbf{Clock cycle:} Also called tick, clock tick, clock period, clock, or cycle. The time for one clock period, usually of the processor clock, which runs at a constant rate.
    
    \item \textbf{Clock period:} The length of each clock cycle.
    
    \item \textbf{Clock cycles per instruction (CPI):} Average number of clock cycles per instruction for a program or program fragment.
    
    \item \textbf{Instruction count:} The number of instructions executed by the program.
    
    \item \textbf{Instruction mix:} A measure of the dynamic frequency of instructions across one or many programs.
    
    \item \textbf{Workload:} A set of programs run on a computer that is either the actual collection of applications run by a user or constructed from real programs to approximate such a mix. A typical workload specifies both the programs and the relative frequencies.
    
    \item \textbf{Benchmark:} A program selected for use in comparing computer performance.
    
    \item \textbf{Million instructions per second (MIPS):} A measurement of program execution speed based on the number of millions of instructions. MIPS is computed as the instruction count divided by the product of the execution time and $10^6$.
\end{enumerate}
\end{document}

